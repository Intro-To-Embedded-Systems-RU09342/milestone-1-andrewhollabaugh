%For these design requirements, really dig into the technical grit. This is where you have to explain what you did as the engineer to meet the system requirements.

The serial communication protocol used is UART (Universal Asynchronous Receiver Transmitter). This protocol allows for easy tranmission of bytes. A standard protocol is used so the nodes can communicate effectively. This protocol can be any number of bytes long, and always starts with a byte containing the number of bytes in the entire transmission. The next three bytes are the brightness values for the first node's LEDs in the order of red, green, then blue. The next sequence of bytes is is the brightness values for next node's LEDs. Any number of multiples of three bytes can be added to this section to add commands for more nodes. The last byte of the transmission is always a newline (0x0d). In the way this protocol is interpreted by the program, the newline character is uncessesary. The program cannot use detecting newlines as a way to detect the end of the transmission, because 0x0d could be a brightness value. Instead, only the first byte (number of bytes in transmission) is used. The program keeps track of how many bytes have been processed to determine the end of the transmission accurately. The program uses an interrupt, which triggers when a UART character is received. When the interrupt is not being run, the system is doing nothing and is set to low-power mode 0 (LPM0).

